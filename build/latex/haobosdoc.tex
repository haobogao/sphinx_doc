%% Generated by Sphinx.
\def\sphinxdocclass{report}
\documentclass[letterpaper,10pt,english]{sphinxmanual}
\ifdefined\pdfpxdimen
   \let\sphinxpxdimen\pdfpxdimen\else\newdimen\sphinxpxdimen
\fi \sphinxpxdimen=.75bp\relax

\PassOptionsToPackage{warn}{textcomp}

\catcode`^^^^00a0\active\protected\def^^^^00a0{\leavevmode\nobreak\ }
\usepackage{cmap}
\usepackage{fontspec}
\usepackage{amsmath,amssymb,amstext}
\usepackage{polyglossia}
\setmainlanguage{english}

\setmainfont{Ubuntu}
\setsansfont{Ubuntu}
\setmonofont{Ubuntu}

\usepackage[Bjornstrup]{fncychap}
\usepackage{sphinx}

\usepackage{geometry}

% Include hyperref last.
\usepackage{hyperref}
% Fix anchor placement for figures with captions.
\usepackage{hypcap}% it must be loaded after hyperref.
% Set up styles of URL: it should be placed after hyperref.
\urlstyle{same}
\addto\captionsenglish{\renewcommand{\contentsname}{Contents:}}

\addto\captionsenglish{\renewcommand{\figurename}{图}}
\addto\captionsenglish{\renewcommand{\tablename}{表}}
\addto\captionsenglish{\renewcommand{\literalblockname}{列表}}

\addto\captionsenglish{\renewcommand{\literalblockcontinuedname}{续上页}}
\addto\captionsenglish{\renewcommand{\literalblockcontinuesname}{continues on next page}}

\def\pageautorefname{页}

\setcounter{tocdepth}{1}


\usepackage{xeCJK}
\usepackage[titles]{tocloft}
\cftsetpnumwidth {1.25cm}\cftsetrmarg{1.5cm}
\setlength{\cftchapnumwidth}{0.75cm}
\setlength{\cftsecindent}{\cftchapnumwidth}
\setlength{\cftsecnumwidth}{1.25cm}


\title{haobo's doc Documentation}
\date{2019 年 09 月 07 日}
\release{1.1}
\author{haobo.gao}
\newcommand{\sphinxlogo}{\vbox{}}
\renewcommand{\releasename}{发布}
\makeindex

\begin{document}

\maketitle
\sphinxtableofcontents
\phantomsection\label{\detokenize{index::doc}}



\chapter{Section Title}
\label{\detokenize{intro:section-title}}\label{\detokenize{intro::doc}}

\section{Section Title}
\label{\detokenize{intro:id1}}

\subsection{Section Title 1}
\label{\detokenize{intro:section-title-1}}

\subsubsection{Section Title}
\label{\detokenize{intro:id2}}

\paragraph{Section Title}
\label{\detokenize{intro:id3}}

\subparagraph{Section Title}
\label{\detokenize{intro:id4}}

\subparagraph{Section Title}
\label{\detokenize{intro:id5}}

\subparagraph{Section Title}
\label{\detokenize{intro:id6}}

\subparagraph{Section Title}
\label{\detokenize{intro:id7}}

\subparagraph{Section Title}
\label{\detokenize{intro:id8}}

\subparagraph{Section Title}
\label{\detokenize{intro:id9}}\begin{itemize}
\item {} 
This is the first bullet list item.  The blank line above the
first list item is required; blank lines between list items
(such as below this paragraph) are optional.

\item {} 
This is the first paragraph in the second item in the list.

This is the second paragraph in the second item in the list.
The blank line above this paragraph is required.  The left edge
of this paragraph lines up with the paragraph above, both
indented relative to the bullet.
\begin{itemize}
\item {} 
This is a sublist.  The bullet lines up with the left edge of
the text blocks above.  A sublist is a new list so requires a
blank line above and below.

\end{itemize}

\item {} 
This is the third item of the main list.

\end{itemize}

This paragraph is not part of the list.
\begin{enumerate}
\item {} 
Item 1 initial text.
\begin{enumerate}
\item {} 
Item 1a.

\item {} 
Item 1b.

\end{enumerate}

\item {} \begin{enumerate}
\item {} 
Item 2a.

\item {} 
Item 2b.

\end{enumerate}

\end{enumerate}
\begin{description}
\item[{term 1}] \leavevmode
Definition 1.

\item[{term 2}] \leavevmode
Definition 2, paragraph 1.

Definition 2, paragraph 2.

\item[{term 3}] \leavevmode{[}classifier{]}
Definition 3.

\item[{term 4}] \leavevmode{[}classifier one{]}{[}classifier two{]}
Definition 4.

\end{description}
\begin{quote}\begin{description}
\item[{Date}] \leavevmode
2001-08-16

\item[{Version}] \leavevmode
1

\item[{Authors}] \leavevmode\begin{itemize}
\item {} 
Me

\item {} 
Myself

\item {} 
I

\end{itemize}

\item[{Indentation}] \leavevmode
Since the field marker may be quite long, the second
and subsequent lines of the field body do not have to line up
with the first line, but they must be indented relative to the
field name marker, and they must line up with each other.

\item[{Parameter i}] \leavevmode
integer

\end{description}\end{quote}
\begin{optionlist}{3cm}
\item [-a]  
Output all.
\item [-b]  
Output both (this description is
quite long).
\item [-c arg]  
Output just arg.
\item [-{-}long]  
Output all day long.
\item [-p]  
This option has two paragraphs in the description.
This is the first.

This is the second.  Blank lines may be omitted between
options (as above) or left in (as here and below).
\item [-{-}very-long-option]  
A VMS-style option.  Note the adjustment for
the required two spaces.
\item [-{-}an-even-longer-option]  
The description can also start on the next line.
\item [-2, -{-}two]  
This option has two variants.
\item [-f FILE, -{-}file=FILE]  
These two options are synonyms; both have
arguments.
\item [/V]  
A VMS/DOS-style option.
\end{optionlist}


\chapter{Indices and tables}
\label{\detokenize{index:indices-and-tables}}\begin{itemize}
\item {} 
\DUrole{xref,std,std-ref}{genindex}

\item {} 
\DUrole{xref,std,std-ref}{modindex}

\item {} 
\DUrole{xref,std,std-ref}{search}

\end{itemize}



\renewcommand{\indexname}{索引}
\footnotesize\raggedright\printindex
\end{document}